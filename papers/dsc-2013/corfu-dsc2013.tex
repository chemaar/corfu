\documentclass{llncs}

%\usepackage{llncsdoc}

%\usepackage{makeidx}  % allows for indexgeneration
\usepackage{graphicx}
\usepackage[T1]{fontenc}
\usepackage[english]{babel}
\usepackage[utf8]{inputenc}
\usepackage{multirow}


\usepackage{rotating}

%%%Math
\usepackage{latexsym}
% \usepackage{amsmath}
% \usepackage{amssymb}
% \usepackage{amsthm}
%\usepackage{eurosans}

\usepackage{eurosym}

\usepackage{longtable}

\usepackage{listings}

\usepackage{color}

\definecolor{darkred}{rgb}{0.5, 0, 0}
\definecolor{violet}{rgb}{1, 0, 1}
\definecolor{green}{rgb}{0.3, 0.95, 0.3}
\definecolor{listinggray}{gray}{0.97}



\begin{document}
\title{A natural language and semantic-based technique to unify corporate names in the e-Procurement sector. The CORFU approach.\thanks{THANKS}}

\titlerunning{}

\author{Jose Mar\'{i}a Alvarez-Rodr\'{i}guez\inst{1}} 

\authorrunning{Jose Mar\'{i}a Alvarez-Rodr\'{i}guez}


\tocauthor{Jose Mar\'{i}a Alvarez}


\institute{The South East European Research Center\\   
  \email{{jmalvarez@seerc.org}},\\
   WWW home page: \texttt{http://www.seerc.org}, \\
}


\date{}

\maketitle

\renewcommand{\labelitemi}{$\bullet$}

\begin{abstract}
Public administrations are currently facing a big challenge trying to improve both the peformance and the transparency of administrative processes.
In this context the e-Government and Open Linked Data initiatives are tackling existing interoperability and 
integration issues among ICT-based systems but the creation of a real transparent environment requires 
much more than the simple publication of data and information in specific open formats; data and information 
quality is the next major step in the pubic sector. More specifically in the e-Procurement domain there is a 
vast amount of valuable data that is already available via the Internet protocols and formats and can be used 
for the creation of new added-value services. Neverthless the simple extraction of statistics or creation of reports 
can imply extra tasks with regards to clean, prepare and reconcile data. 
From a transparency point of view one of the most interesting services lies in tracking rewarded contracts (type and supplier). 
Depending on the capabilities of the public organization this kind of basic report can turn into a 
complex task due to a lack of standardization in supplier names or the use of different descriptors for the type of contract. That is why 
this paper presents a stepwise method based on natural language processing and semantics to deal with the hetereogenities in corporate names. 
Furthermore a research study to evaluate the precision and recall of the proposed technique, using as use case the public dataset of rewarded public 
contracts in Australia during the period 2005-2012, is also presented, finally some discussion, conclusions and future work 
are also outlined.
\end{abstract}

\section{Introduction}
Public bodies are continuously publishing procurement opportunities in which 
valuable metadata is available. Depending on the stage of the process new data arises such 
as the supplier name that has been rewarded with the public contract. In this 
context the extraction of statistics on how many contracts have been 
rewarded to the same company is a relevant indicator to evaluate the transparency 
of the whole process. Although companies that want to tender for a public contract must be 
oficially registered and have an unique identification number, the truth is 
that most of rewarded contracts the supplier is only identified by a name or a string literal typed 
by a civil-servant. In this sense there is not usually a connection between 
the oficial company registry and the process of rewarding contracts implying different 
naming problems and inconsistenty in data that are spread to next stages preventing future 
activities such as reporting.

Organization, corporate, company or institution to name a few (hereafter these names will be used imply 
that the task of grouping contracts by a supplier is not a mere process of searching by the same literal. 
Technical issues such as hyphenation, use of abbreviations or acronynms an transliteration are 
common problems that must be addressed in order to provide a final corporate name and have been widely studied in the 
field of Name Entity Recognition~\cite{citeulike:1657521} (NER) or name entity disambiguation~\cite{Sarmento:2009:AWN:1602022.1602085,Klein:2003:NER:1119176.1119204}. 
Neverthless the problem that is being tackled in these approaches lies in the identification of organization 
names in a raw text while in the e-Procurement sector the string literal identifying a supplier is already 
known and it is not necessary to extract or recognize a potential name.


In the particular case of Australia e-Procurement domain, the supplier name is introduced by typing a string literal without any assistance or 
auto-complete method. Obviously a variety of errors and variants for the same company, see Table~\ref{tabla:aus-suppliers}, 
can be found such as misspelling errors~\cite{NorvigSpelling,StanfordSpelling}, name and acronym mismatches~\cite{Yeates99automaticextraction,Ratinov:2004:AES:1025132.1026366} 
or context-aware data that is already known when the dataset is processed such as the country. Furthermore it is also well-known that a large company can 
be divided into several divisions or departments but from a statistical point of view grouping data by a supplier name 
should take into account all rewarded contracts regardless the structure of the company.


On the other hand the application of semantic technologies and the Linking Open Data approach (hereafter LOD)~\cite{Berners-Lee-2006,Heath_Bizer_2011}  
in several fields like e-Government (e.g. the Open Government Data initiative) tries to improve the knowledge about a specific area providing 
common data models and formats to share information and data between agents. More specifically, in the European e-Procurement 
context~\cite{e-Proc-map-paper} there is an increasing commitment to boost the use of electronic communications and transactions 
processing by government institutions and other public sector organizations in order to provide added-value services with special focus on SMEs. 
More specifically the LOD initiative seeks for creating a public data realm in which one the principles of this initiative that lies in the 
unique identification or resources through URIs can become real. Thus entity reconciliation techniques~\cite{Serimi,conf/www/MaaliCP11} 
coming from the ontology mapping and alignment areas or algorithms based on Natural Language Processing (hereafter NLP) have been 
designed to link similar resources already available in different vocabularies, datasets or databases such as DBPedia or Freebase. 
In the specific case of company names there is an open database, OpenCorporates, that has collected more than $52$ million of names 
around the world and it can be considered a perfect candidate to perform the reconciliation process between a string literal and a 
target resource to obtain an unique identifier. Nevertheless the issue of unifying supplier names as a human would do 
faces new  problems that have been tackled in other research works~\cite{Galvez2006} to extract statistics of performance in bibliographic databases. 
The main objective is not just a mere reconciliation process to link to existing resource but to create a unique name or link. For instance in the case 
of the ongoing example the string literals ``Oracle'' and ``Oracle University'' could be respectively aligned to the entity $<$Oracle$>$ and $<$Oracle\_University$>$ but 
the problem of grouping by a unique (\textit{Big}) name or resource still remains. 

That is why a context-aware method based on NLP techniques combined with semantics has been designed, customized and implemented trying 
to exploit the naming convention of a specific dataset.

The remainder of this paper is structured as follows. Section 2 a literature review is presented. Afterwards next section outlines main mismatches in corporate names. Section 4 presents 
the CORFU approach to unify corporate names. The evaluation section exposes and discusses the experimentation carried out to test the presented approach using as a dataset the rewarded 
contracts of Australia in the period 2005-2012. Finally conclusions summarizes the main outcomes of this work 
and some open issues are also presented as future work.



\begin{table}[!htb]
\renewcommand{\arraystretch}{1.3}
\begin{center}
\begin{tabular}{|l|l|}
\hline
  \textbf{Raw Supplier Name} & \textbf{Target (potential) Supplier Name}\\  \hline
  Accenture & \multirow{12}{*}{Accenture} \\
  ACCENTURE & \\ 
  Accenture Aust Holdings & \\  
  Accenture Aust Holdings Pty Ltd & \\
  Accenture Australia & \\
  ACCENTURE AUSTRALIA & \\
  Accenture  Australia Holding P/L & \\
  Accenture Australia Holdings P/Ltd & \\
  Accenture Australia Holdings Pty & \\
  Accenture Australia Holdings Pty Lt  & \\  
  Accenture Australia Limited & \\
  \ldots  & \\
  Accenture Australia Ltd & \\ \hline
  Microsoft Australia &  \multirow{3}{*}{Microsoft} \\
  Microsoft Australia Pty Ltd & \\
  \ldots  & \\
  Microsoft Enterprise Services & \\ \hline
  Oracle (Corp) Aust Pty Ltd  & \multirow{14}{*}{Oracle} \\
  Oracle Corp (Aust) Pty Ltd  & \\
  Oracle Corp Aust Pty Ltd & \\
  ORACLE CORP AUST PTY LTD & \\
  Oracle Corp. Australia & \\
  ORACLE CORP AUSTRALIA P/L & \\
  Oracle Corp. Australia Pty.Ltd. & \\
  ORACLE CORP AUSTRALIA PTY LTD & \\
  Oracle Corpoartion (Aust) Pty Ltd & \\
  Oracle Corporate Aust Pty Ltd & \\
  Oracle Corporation & \\
  Oracle Risk Consultants & \\
  ORACLE SYSTEMS (AUSTRALIA) PTY LTD & \\
  \ldots  & \\
  Oracle University  & \\ \hline
  PRICEWATERHOUSECOOPERS(PWC)  & \multirow{8}{*}{PricewaterhouseCoopers} \\
  PricewaterhouseCoopers Securities Ltd & \\
  PriceWaterhouseCoopers Securities Ltd & \\
  PRICEWATERHOUSE COOPERS SECURITIES LTD & \\
  PricewaterhouseCoopers Services LLP & \\
  Pricewaterhousecoopers Services Pty Ltd & \\
  PriceWaterhouseCoopers (T/A: PriceWaterhouseCoopers Legal) & \\
  \ldots  & \\
  Pricewaterhouse (PWC) & \\ \hline
  \ldots & \ldots \\
  \hline
  \end{tabular}
  \caption{Examples of supplier names in Australian rewarded contracts.}
  \label{tabla:aus-suppliers}
  \end{center}
\end{table} 


\section{Related Work}
The literature review of this review covers the following areas:
\begin{itemize}
 \item Natural Language Processing and Computational Linguistics. In these research areas common works dealing with the aforementioned data hetereogenities 
   such as misspelling errors~\cite{NorvigSpelling,StanfordSpelling} and name/acronym mismatches~\cite{Yeates99automaticextraction,Ratinov:2004:AES:1025132.1026366}, 
  in the lexical, syntactic and semantic level can be found. These approaches can be applied to solve general problems and usuarlly follow a 
  traditional approach of text normalization, lexical analysis, pos-tagging word according to a grammar and semantic analysis to filter or 
  provide some kind of service such as information/knowledge extraction, reporting, sentiment analysis or opinion mining. 
  Well-stablished APIs such as NLTK~\cite{LoperBird02} for Python, Lingpipe~\cite{Lingpipe}, OpenNLP~\cite{OpenNLP} or Gate~\cite{Gate} for Java, WEKA~\cite{read12:_scalab,} 
  (a data mining library with NLP capabilities), the Apache Lucene and Solr~\cite{rafa2011apache} search engines provide the proper building blocks to build natural-language based applications. Recent times have seen how the analysis 
  of social networks such as Twitter~\cite{Li:2012:TNE:2348283.2348380,Gimpel:2011:PTT:2002736.2002747}, the extraction of 
  clinical terms~\cite{Wang:2009:ARN:1667884.1667888} for electronic health records, the creation of bibliometrics~\cite{Galvez2006,Morillo:2013:TAA:2424697.2424727} or 
  the idenfication of gene names~\cite{Krauthammer:2004:TIB:1053007.1053018,Galvez2012} to name a few have tackled the problem of entity recognition and extraction from raw sources. 
  Other supervised techniques~\cite{Bohn:2006:PHD} have also be used to train data mining-based algorithms with the aim of creation 
  classifiers.
 
 \item Semantic Web. More specifically in the LOD initiative the use of entity reconciliation techniques to uniquely identify resources 
 is being currently explored. Thus an entity reconciliation pocess can be briefly defined as the method for looking and mapping two different 
 concepts or entities under a certain threshold. There are a lot of  works presenting solutions about concept mapping, entity reconciliation, etc. 
 most of them are focused on the previous NLP techniques~\cite{conf/www/MaaliCP11,Serimi} (if two concepts have similar literal descriptions then they should be similar) 
 and others (ontology-based) that also exploit the semantic information (hierarchy, number and type of relations) to establish a potential mapping 
 (if two concepts share similar properties and similar super classes then these concepts should be similar). Apart from that 
 there are also machine learning techniques to deal with these mismatches in descriptions using statistical approaches. Recent times, 
 this process has been widely studied and applied to the field of linking entities in the LOD realm. Although there is no way of automatically 
 creating a mapping with a 100\% of confidence (without human validation) a mapping under a certain percentage of confidence can be 
 enough for most of user-based services such as visualization. However, in case of using these techniques as previous step of a reasoning or 
 a formal verification process this ambiguity can lead to infer incorrect facts and must be avoided without a previous human validation. 

 On the other hand the use of semantics is also being applied to model organizational structures. In this case the notion 
 of corporate is presented in several vocabularies and ontologies as Dave Reynolds (Epimorphics Ltd) reports~\footnote{\url{http://www.epimorphics.com/web/wiki/organization-ontology-survey}}. 
 Currently the main effort is focused in the designed of the Organizations Vocabulary (a W3C Working Draft) in which the structure and 
 relationships of companies are being modelled but some problems are emerging due to some factors: 1) missing pieces to define the status of the organization; 
 2) tangled parts to specify the structure (concepts and relations) between the elements of the organization; 
 3) lack of text properties and, in our research study, 4) name and address mismatches. This proposal is especially relevant in some senses to: 
 1) unify existing models to provide a common specification; 2) apply semantic web technologies and the Linked Data approach to enrich 
 and publish the relevant corporate information; 3) provide access to the information via standard protocols 
 and 4) offer new services that can exploit this information to trace the evolution and behavior of the organization over time.

 
 \item Corporate Databases. Figure~\ref{figure:open} shows an example of an organization (in N3 format) using ``The Open Database Of The Corporate World''~\footnote{\url{http://opencorporates.com/}}. 
 This information is potentially relevant to this work due to the large database that Open Corporates provides ($54,080,317$ of companies in May 2012) 
 with high-valuable information like the company ID. These data follow a mixed approach between Open and the LOD approach but a formal 
 model describing the organizations is missing. That is why the use of a common ontology could improve the information sharing and 
 the exploitation of the information in a standard way generating new value-added services of five stars like activity tracking. 

\begin{figure}[!h]
\begin{center}
\begin{lstlisting}[language=SPARQL]
...
<http://opencorporates.com/id/companies/us_az/F07503757#id> 
	dct:created "1995-06-01"^^xsd:date;
	a adms:Identifer;
	skos:notation "F07503757";
	adms:schemaAgency "Arizona Corporation Commission".

<http://opencorporates.com/id/companies/us_az/F07503757#ra> 
	a locn:Address;
	locn:fullAddress "% CORPORATION SERVICE COMPANY, 
	2338 W ROYAL PALM RD STE-J, PHOENIX, AZ 85021".

<http://opencorporates.com/id/companies/us_az/F07503757> 
	opencorporates:companyType "CORPORATION";
	opencorporates:legalName "ORACLE SOFTWARE CORPORATION (FN)";
	a <http://s.opencalais.com/1/type/er/Company>,
		legal:LegalEntity;
	rdfs:label "ORACLE SOFTWARE CORPORATION (FN)";
	vCard:adr _:bnode1324364416;
	legal:companyType "CORPORATION";
	legal:legalIdentifier 
	  <http://opencorporates.com/id/companies/us_az/F07503757#id>;
	legal:legalName "ORACLE SOFTWARE CORPORATION (FN)";
	locn:registeredAddress 
	  <http://opencorporates.com/id/companies/us_az/F07503757#ra>.
...
\end{lstlisting}
\caption{Partial Information in the N3 format about an ``Oracle'' company in ``Open Corporates''.}
\label{figure:open}
\end{center}
\end{figure}
 
 
\end{itemize}


\section{Mismatches in Corporate Names}
According to~\cite{Galvez2006,Morillo:2013:TAA:2424697.2424727} some name variations or mismatches can be found:

\begin{itemize}
 \item Non-acceptable variations, including non-valid names, or incorrect
variant forms. The reason behind such variations would essentially be
errors, misspelled words and inaccurate translations of foreign terms.

\item Acceptable variations, which would be valid names, or correct variant
forms. Here the most frequent causes are permuted word order or distinct
syntactic formats of the same name, the splitting of words, acronyms, full
vs. abbreviated address, transliteration differences, multilingual character 
of information (US, UK or AUS English), the inclusion or exclusion of geoinformation, 
inclusion or not of the main suborganization or subsidaries. Overall, these variations are 
interchangeable in specific contexts without leading to a change in meaning.
\end{itemize}



% 
% \subsection{Modeling Organizational Structures}
% The broad objective of modeling organizational structures is to promote this information using semantic technologies and the LOD approach. To get this objective 
% the Organizations Ontology~\footnote{\url{http://www.epimorphics.com/public/vocabulary/org.html}} represents a first step to model organizations but 
% some issues should be addressed to spread the scope of this specification: 1) Structure; 2) Human resources; 3) Corporate image; 4) Id; 5) Name; 
% 6) Purposes and intentions; 7) Cataloging products, services and activities; 8) Multilingual and multicultural problems; 9) Inter/Intra relationships or 
% 10) Activity Tracking and Financial transactions (e.g XBRL could be used). Nevertheless we have used this ontology to initially address the objectives of this work because 
% it is core ontology for organizational structures (see Fig\footnote{Source: http://www.epimorphics.com/public/vocabulary/diagram.png}.~\ref{fig:org}), 
% aimed at supporting linked-data publishing of organizational information across a number of domains. It is also designed 
% to allow domain-specific extensions to add classification of organizations and roles, 
% as well as extensions to support neighboring information such as organizational activities. 
% This ontology partially fits to the aim of modeling organizations in a standard and reusable way 
% with semantic technologies. 
% 
% 
 \section{The CORFU approach}
 
The technique to generate a unique name before performing the reconciliation process is a stepwise method, see 1, in which each step performs a filter over 
the string literal trying to remove all unnecessary words in the name to finally use an iterative process of string comparison and grouping to generate a unique 
and relevant name for the input dataset. This process has been implemented using the NLTK library and other external Python APIs such as fuzzywuzzy or a spell 
checker based on the well-known Peter Norvig speller. After this initial process of unifying names a second step to reconcile names can be easily done reusing 
resources in OpenCorporates, DBPedia, Linkedin Companies or Google Places. Although a naïve implementation is already available it is considered an 
extension and ongoing work that is out of the scope of this report.  
% 
\section{Evaluation}

\subsection{Research design}
% Since this recommender engine is designed to be used by professionals 
% in the journalism domain rather than regular users the purpose of this study 
% is to compare a set of news suggested by the recommendation engine with the 
% expected results of a panel of experts. Thus, the objective is 
% to asses if recommendations provided by Wesomender can fulfill 
% the expectations and requirements of professionals in this sector. 
% 
\subsection{Sample}
In order to validate the approach outlined in this summary the dataset of supplier names in Australia in the period 2005-2012 containing $430188$ full names 
and $77526$ unique names has been selected. Initially the traditional reconciliation process using 
Google Refine and OpenCorporates generated an 8\% of links but most of them were 
incorrect or not grouped by the same resource. After applying the above-mentioned s
tepwise method the initial set $77526$ names were grouped in $40278$ 
distinct names (51\% of potential right links to OpenCoporates). 
Furthermore these unified names were reviewed by hand and in the specific case 
of the first one hundred companies in the Forbes list a 100\% of correct names can be now ensured.
 
\subsection{Results and Discussion}
The main conclusion of this works lies in the design of a technique to prepare 
raw organization names in a specific context, e.g. Australia supplier names,  
before performing a reconciliation process. Although the percentage of potential 
right links to existing datasets has been dramatically improved it is clear that 
human-validation is also required to ensure the correct unification of names. 
Other NLP techniques based on n-grams or a classifier will be used in the future 
to deliver a complete and intelligent company unifier. On the other hand, the 
application of this technique enables the comparison of rewarded contracts to 
different companies and can help to improve the transparency in public 
administrations.

\section{Conclusions and Future Work}

% % Now the information about organizations is considered to be a key factor for the transparency 
% % and the improvement of corporate image of companies. In that sense public administrations are 
% % very interested in the publication of their data following the LOD approach (e.g. using RDFa) and the Organizations Ontology 
% % (it was original motivated by a need to publish information relating to government organizational 
% % structure as part of the data.gov.uk initiative) is a first step to reach this broad objective. That is why a 
% % specification to model organizations can change the current approach to discover, activity track and search organizations 
% % in a specific domain. For instance in the e-Procurement sector an organization can be tracked 
% % making possible the extraction of statistics about their public contracts (type, region, etc.). 
% % In this document a review of existing ontologies for organizations is presented and an application of the Organizations Ontology 
% % is also outlined to make the profile of an organization that wants to tender in the e-procurement sector.
% %  This document is considered to be the first step to work in a common specification for modeling and exploiting
% % information about organizations in the new realm of Linked Data. Regarding the future work, the results of this study are intended to be
% % exploited by a commercial service like Eurolert.net~\cite{web20} and we are also interested in reporting the 
% % results to \textit{The Internal Market and Services Directorate General (DG MARKT) of the European Commision},  
% % \textit{The Information Society and Media Directorate General (DG INFSO) of the European Commision}, 
% % the LOD and OGD initiatives among others.
% 


% 
% \section{Acknowledgements}
% 
% % 
% % \section{Introduction}
% % An organization is considered to be a set of constraints on the activities performed by agents.  
% % This approach was presented in~\cite{Weber1978} who viewed the process of bureaucratization as a shift 
% % from management. Previously, Mintzberg provided an analysis of organization structure 
% % distinguishing among five basic parts of an organization and five distinct organization configurations. 
% % This vision put together some mechanisms to achieve coordination with the objective of modeling goals, 
% % business processes and rules, authority, positions and communication. In this context, some works 
% % are emerged~\cite{Fox95anorganisation} trying to create ontologies and models that specify the realm of organizations but 
% % different problems are arisen due to some factors: 1) missing pieces to define the status of the organization; 
% % 2) tangled parts to specify the structure (concepts and relations) between the elements of the organization; 
% % 3) lack of text properties, 4) name mismatches, etc. 
% % 
% % Currently the application of semantic technologies and the Linking Open Data approach (hereafter LOD)~\cite{heath11linked} in several fields 
% % like e-Government (e.g. Open Government Data initiative) tries to improve the knowledge about a specific area providing 
% % common data models and formats to share information and data between agents. More specifically, 
% % in the European e-procurement context there is an increasing commitment to boost the use 
% % of electronic communications and transactions processing by government institutions and other public sector organizations 
% % in order to provide added-value services with special focus on SMEs. In that sense modeling organizational structures
% % with these techniques can help to fulfill the requirements of an innovative unified e-procurement pan-European information 
% % system in which the information about organizations (structure, human resources, corporate image, id, address, name, purposes, products and 
% % services, activities with others, etc.) plays a key-role to match (and track) organization intentions (and activities) 
% % with public procurement notices. Putting together these facts the following example partially 
% % motivates this work with the objective of reusing information about organizations.
% % 
% % E.g: \textit{Which public procurement notices are relevant to Dutch companies (only SMEs) that want to tender for
% % contracts announced by local authorities with a total value lower than 170K \euro\mbox{ }to procure
% % ``Construction work for bridges and tunnels, shafts and subways`` and a two year duration in the 
% % Dutch-speaking region of Flanders (Belgium)?}.
% % 
% % Finally, this work aims to apply the aforementioned techniques to model and ease the access 
% % to organization’s information addressing specifically the principles of OGD\footnote{\url{http://resource.org/8\_principles.html}}. Following, 
% % the main contributions are highlighted: 1) extract the public information available about organizations and 
% % unify existing models to provide a common specification; 2) apply semantic web technologies and the LOD approach; 
% % 3) provide access to the information via standard protocols and 4) offer new services that can exploit 
% % this information to trace the evolution and behavior of the organization over time.
% % 
% % \section{Related work}
% % In the case of organizations it is quite a hard search to do because a lot of 
% % ontologies need some notion of Organization to point to. E.g. FOAF is about people but needs to mention the ``Organizations'' 
% % of which a person is a member, Inference Web~\footnote{\url{http://inference-web.org/}} is about distributed 
% % inference but covers trust and provenance which in turn requires a notion of organizations 
% % (that are in a trust relationship). In each of these cases the representation of Organization is minimal. 
% % Following the evaluation made by Dave Reynolds (Epimorphics Ltd) in this web report\footnote{\url{http://www.epimorphics.com/web/wiki/organization-ontology-survey}}, several approaches can be found. 
% % Firstly, in previous works there are a number of similar upper ontologies (Proton, Sumo, SmartWeb) that include 
% % some notion of Organization. These models have a lot of other intentions that are not match with 
% % the specific requirement of a small and reusable model to describe organizations 
% % but they should be reviewed. Secondly, if search engines are used to look for the concept of 
% % Organization the next results will be found: Swoogle $3,990$ matches, Falcons gives $15,881$ hits for Organization concept 
% % from $15$ highlighted vocabularies and Google turns up: 1) the ``Organization Ontology 1.0'' written in SHOE, 
% % giving a basic hierarchy of organization, industries and employee roles; 2) an ``Organization Ontology for Enterprise Modelling'' 
% % which is focused on supply chain and 3) ``Enterprise Ontology'', an ontology to represent the activity of
% %  business enterprises expressed in Ontolingua. Jeni Tennison~\footnote{\url{http://www.jenitennison.com/}} has also pointed 
% % to an ontology developed by TSO for the London Gazette RDFa markup: Gazette Organization and Gazette Person. 
% % According to the authors of this survey the following approaches should be reviewed: AKT Portal Ontology,
% % Proton top, Good Relations, FOAF, SIOC, Enterprise Modelling Ontology, Enterprise Ontology, Gazette organization and 
% % person ontologies, Provenance Vocabulary Core Ontology, and other like School of ECS (University of Southampton) or 
% % Academic Institution Internal Structure Ontology (AIISO), vocabularies for describing the internal organizational 
% % structure of an academic institution. This situation implies a tangled environment for describing organizations and supposes
% % a barrier to promote this information to the new Web of Data.
% % 
% % In the scope of LOD and open government data (OGD) there are projects
% % trying to exploit the published information in some domains like 
% % LOTED~\footnote{\url{http://loted.eu:8081/LOTED1Rep/}} (``Linked Open Tenders
% % Electronic Daily'') in the e-procurement sector  where they use the RSS feeds of TED. 
% % UK government\footnote{\url{http://data.gov.uk}} is doing a great effort to
% % promote its information sources using the LOD approach. They have published datasets
% % from different sectors: transport, defense, NUTS geographical
% % information~\footnote{\url{http://nuts.psi.enakting.org/}}, etc. Most of the
% % public administrations in the different countries are also betting for LOD
% % approach to make public their information: Spain (Aporta
% % project~\footnote{\url{http://www.aporta.es/}}),
% % USA\footnote{\url{http://www.data.gov/}}, etc. On the other hand, Product Scheme Classifications (also known as PSCs) 
% % like the CPV (Common Procurement Vocabulary available at RAMON, the Eurostat's metadata server) have been built to solve
% % specific problems of interoperability and communication in e-commerce\cite{Volker02amodeling,Corcho01solvingintegration}. 
% % The aim of a PSC is to be used as a standard \textit{de facto} by different agents for information interchange 
% % in marketplaces~\cite{DBLP:journals/tcci/Alor-HernandezAJPRMBG10}. Any PSC, as well as other classification systems can 
% % be interpreted as: 1) domain-ontologies~\cite{Hepp-possible} or 2) conceptual schemes~\cite{chemaEurovoc2008} comprised 
% % of conceptual resources . Finally, Good Relations is an ontology for the e-commerce developed by 
% % Martin Hepp et. al and now integrated the Yahoo Real Estate portal via RDFa.
% % 
% % In the field of the semantic web technologies and for modeling the domain knowledge 
% % there are several options: RDF, RDF(S), OWL 2 or SKOS among others. 
% % They provide a common format and data model for sharing and linking knowledge organization systems 
% % via the web. This information can be retrieved using SPARQL, 
% % a query language and a protocol to retrieve the information of datasets published via an endpoint. 
% % Currently, there is a working group defining a vocabulary and a set of instructions 
% % that ease the discovery and usage of linked datasets (voID~\footnote{\url{http://vocab.deri.ie/void/guide}}), 
% % the new specification of SPARQL (1.1~\footnote{\url{http://www.w3.org/TR/sparql11-service-description/}}) 
% % enables a method for discovering and vocabulary for describing SPARQL services 
% % made available via an endpoint and Pubby\footnote{\url{http://www4.wiwiss.fu-berlin.de/pubby/}} or 
% % ELDA~\footnote{\url{http://code.google.com/p/elda/}} are implementations of a linked data frontend that provide a 
% % configurable way to access RDF data using simple RESTful URLs that are translated into 
% % queries to a SPARQL endpoint. 
% % 
% % \subsection{Previous Work: Open Corporates}\label{open-corporates}
% % 
% %
% % \section{Modeling Organizational Structures}
% % The broad objective of modeling organizational structures is to promote this information 
% % using semantic technologies and the LOD approach. To get this objective 
% % the aforementioned Organizations Ontology~\footnote{\url{http://www.epimorphics.com/public/vocabulary/org.html}} represents a 
% % first step to model organizations but some issues should be addressed to spread the scope of this specification: 
% % 1) Structure; 2) Human resources; 3) Corporate image; 4) Id; 5) Name; 6) Purposes and intentions; 7) Cataloging products, services and activities;
% %  8) Multilingual and multicultural problems; 9) Inter/Intra relationships or 10) Activity Tracking and Financial transactions 
% % (e.g XBRL could be used). Nevertheless we have used this ontology to initially address the objectives of this work because 
% % it is core ontology for organizational structures (see Fig\footnote{Source: http://www.epimorphics.com/public/vocabulary/diagram.png}.~\ref{fig:org}), 
% % aimed at supporting linked-data publishing of organizational information across a number of domains. It is also designed 
% % to allow domain-specific extensions to add classification of organizations and roles, 
% % as well as extensions to support neighboring information such as organizational activities. 
% % This ontology partially fits to the aim of modeling organizations in a standard and reusable way 
% % with semantic technologies. 
% % 
% % \begin{figure}[h]
% %  \centering
% % %    \includegraphics[width=10cm]{images/org}
% %     \caption{Organizations Ontology. Overview.}
% %  \label{fig:org}
% % \end{figure}
% % 
% % In order to accomplish with the contributions of this work we should define:
% % \begin{itemize}
% %  \item the process of extracting (and structuring) the public information available about organizations. In this case, we have developed
% % a system\footnote{We have discarded the use of Google Refine and similar tools due to the huge amount of data to be processed.} 
% % that takes the organizations~\footnote{\url{ftp://ftp.ted.europa.eu/META-XML/}} in CSV format available at TED 
% %  ($\sim$ 320K) and transforms this information (ID, address, contact person and CPV codes or purposes inferred by the previous public 
% % contracts awarded) to RDF according to the Organizations Ontology, the principles of LOD~\cite{heath11linked} paying special 
% % attention to the URL design and SKOS to deal with multilingual issues. After that all generated triples are stored 
% % in the triple store OpenLink Virtuoso~\footnote{\url{http://virtuoso.openlinksw.com/}} (providing a SPARQL endpoint). 
% % \item the access to the published information via Pubby. This linked data frontend is used to access generated triples but we have also developed a gateway that translates, on demand, the data of an ``OpenCorporates
% % Organization'' (see Sect.\ref{open-corporates}) to our system unifying the access to the information through Pubby without replication of information.
% % \item the exploitation of this information. Currently it is mainly made by two services: 1) an enhanced matchmaking service to search public procurement notices and
% % 2) a simple organization activity track service in the e-procurement sector.
% % \end{itemize}
% % 
% % \section{Use Case: Matchmaking and Tracking Organizations in the E-procurement sector}
% % In the e-Procurement information domain, one of the targeted services~\cite{metteg2011} to be improved is the 
% % ``search of public procurement notices according to a profile''. In the context of searching, 
% % matchmaking refers to the procedure of retrieving a relevant list of results that matches with the 
% % intentions of an organization that wants to tender in a specific activity sector. Other interesting service 
% % on e-Procurement is the extraction of statistics to generate reports about the history of 
% % some place, organization or contracting authority that can be exploited through temporal series, 
% % weighted aggregation operators or statistical inference, specifically predictive inference. 
% % 
% % Let be $E$ an organization that wants to tender in a public procurement process, 
% % the representation using N3, this information is provided the process of extracting (and structuring) 
% % the public information available about organizations, is presented in Figure~\ref{figure:org}:
% % 
% % \begin{figure}[p]
% % \begin{center}
% % \begin{lstlisting}[language=SPARQL]
% %  <http://mydutchcompany.com/> a v:VCard ;
% %      v:fn "Dutch Company Inc." ;
% %      v:org [   v:organisation-name "Dutch Company Inc." ;
% %              v:organisation-unit "Corporate Division" ] ;
% %      v:adr [ rdf:type v:Work ;
% %              v:country-name "Netherlands" ;
% %              v:locality "Amsterdam" ;
% %              v:postal-code "1016 XJ" ;
% %              v:street-address "Lijnbaansgracht 215" ] ;
% %      v:geo [ v:latitude "52.36764" ;
% %              v:longitude "4.87934" ] ;
% %      v:tel [ rdf:type v:Fax, v:Work ;
% %              rdf:value " +31 (10) 400 48 00"] ; 
% %      v:email <mailto:company@mydutchcompany> ;
% %      v:logo <http://mydutchcompany.com/logo.png> .
% % 
% % <http://purl.org/weso/units/euro> a muo:UnitOfMeasurement;
% %     muo:measuresQuality <http://purl.org/weso/physicalQuality/Money>.
% %     muo:altSymbol "\eur" ;
% %     muo:prefSymbol "\eur" .
% % 
% % <http://purl.org/weso/ppn/noticeValue>  a muo:QualityValue;
% %     muo:numericalValue "170.000";
% %     muo:inTime "2011-01-12" ;
% %     muo:measuredIn <http://purl.org/weso/units/euro>.
% % 
% % <http://purl.org/weso/units/year> a muo:UnitOfMeasurement;
% %     muo:measuresQuality <http://purl.org/weso/physicalQuality/Time>.
% %     muo:altSymbol "year" ;
% % muo:prefSymbol year" .
% % 
% % <http://purl.org/weso/ppn/noticeDuration>  a muo:QualityValue;
% %     muo:numericalValue "2";
% %     muo:inTime "2011-01-12" ;
% %     muo:measuredIn <http://purl.org/weso/units/year>.
% % 
% % <http://purl.org/weso/organizations#dutchOrganization> a org:FormalOrganization;
% %     org:purpose cpv:45221000;
% %     org:purpose cpv:45221113 ; 
% %     org:purpose  <http://purl.org/weso/ppn/noticeValue> ; 
% %     org:purpose  <http://purl.org/weso/ppn/noticeDuration> ; 
% %     org:purpose <http://sws.geonames.org/50.85_43.49/ > ;
% %     skosxl:prefLabel "Dutch organization" ;
% %     org:classification <http://purl.org/organizations#SME>;
% %     org:hasSite <http://mydutchcompany.com/> ;
% %     org:siteAddress <http://mydutchcompany.com/> ;
% %     ... 
% % \end{lstlisting}
% % \caption{Information about an organization in N3.}
% % \label{figure:org}
% % \end{center}
% % \end{figure}
% % 
% % Following the input SPARQL query of the motivating example including the profile of an organization, a CPV code, a NUTS region 
% % (only coordinates) and some numeric values for total value and duration is presented, see Fig.~\ref{figure:simple}.
% % 
% % \begin{figure}[!h]
% % \begin{center}
% % \begin{lstlisting}[language=SPARQL]
% % SELECT * WHERE{
% %   ?notice rdf:type ppn:PublicProcurementNotice .
% %   ?notice dct:identifier ?id .
% %   ?notice dct:description ?description .
% %   ?notice ppn:hasStatus ppn:Active .
% %   ?notice org:classification <http://purl.org/organizations#SME> .
% %   ?notice wgs84_pos:lat ?lat.
% %   ?notice wgs84_pos:lon ?long .
% %   ?notice ppn:totalValue ?totalValue.
% %   ?amount muo:measuredIn <http://purl.org/weso/units/euro> .
% %   ?notice ppn:duration ?duration.
% %   ?duration muo:measuredIn <http://purl.org/weso/units/year> .
% %  FILTER ( 
% %  ((?notice ppn:hasCPVcode cpv:45221000))
% %  and (?lat == "50.85") and (?long == "43.49")
% %  and (?totalValue <= 170,000^xsd:double) and (?duration <= 2) )}
% % \end{lstlisting}
% % \caption{Simple SPARQL query according to Organization profile.}
% % \label{figure:simple}
% % \end{center}
% % \end{figure}
% % 
% % After the process of query expansion a new SPARQL query\footnote{The URI prefixes of this example come from the ``Prefix.cc'' service. 
% % Muo and ppn (namespaces for ``Units of measurement ontology`` and ``Public Procurement Notices'') prefixes are also added to do a more human-readable example.} is built, 
% % see Fig.~\ref{figure:expanded}. The process of expansion, using techniques like Spreading Activation, algorithms in the Apache Mahout library or others like~\cite{citeulike:9135863}, 
% % selects new CPV codes (45221100-``Construction work for bridges'', 45221110-``Bridge construction work'', 45221111-``Road bridge construction work'', 45221113-``Footbridge construction work''), 
% % new NUTS codes (spreading the geographical scope) and establish a range for the numeric
% % variables according to the historical information available at the database. 
% % \begin{figure}[!h] 
% % \begin{center}
% % \begin{lstlisting}[language=SPARQL]
% % SELECT * WHERE{
% %   ...
% %   ?notice nuts:containedBy ?place .
% % FILTER (  ( (?notice ppn:hasCPVcode cpv:45221000) or 
% % 	    (?notice ppn:hasCPVcode cpv:45221110) or
% % 	    (?notice ppn:hasCPVcode cpv:45221111)...)
% % 	  ( (?place nuts:containedBy nuts:NUTS-NL326 ) or 
% % 	    (?place nuts:containedBy nuts:NUTS-1025) or 
% % 	    (?place nuts:containedBy nuts:NUTS-B3) or
% % 	    (?place nuts:containedBy nuts:NUTS-BE2) or ...) 
% % 	and (?duration > 2 and ?duration <= 3) 
% % 	and (?totalValue > 130,000^xsd:double 
% % 	     and ?totalValue <= 200,000^xsd:double))}
% % \end{lstlisting}
% % \caption{Expanded SPARQL query.}
% % \label{figure:expanded}
% % \end{center}
% % \end{figure}
% % 
% % The relevance of this example lies on the demonstration that the information about organizations
% % is a key-factor to provide enhanced services in different domains. In this case the e-procurement
% % sector has been selected due to the ongoing research of the \#SKIP\# project. A public demo is also available
% % at \#SKIP\#~\footnote{\url{\#SKIP\#}} and the data can be easily retrieved querying the SPARQL endpoint of
% % the organizations dataset. Further implementations are supposed to exploit this data in order to provide more advanced
% % statistics service, identify organizations across the web (resolving name mismatches) and to track corporate
% % activities in different sectors.
% % \section{Conclusions and Future Work}

\bibliographystyle{plain}
% %\bibliographystyle{unsrt}
% %\bibliographystyle{acm}
\bibliography{references}
% \renewcommand{\bibname}{References}
\end{document}

